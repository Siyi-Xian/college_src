%%%%%%%%%%%%%%%%%%%%%%%%%%%%%%%%%%%%%%%%%%%%%%%%%%%%%%%%%%%%%%%%%%%%%
% LaTeX Template: Project Titlepage Modified (v 0.1) by rcx
%
% Original Source: http://www.howtotex.com
% Date: February 2014
% 
% This is a title page template which be used for articles & reports.
% 
% This is the modified version of the original Latex template from
% aforementioned website.
% 
%%%%%%%%%%%%%%%%%%%%%%%%%%%%%%%%%%%%%%%%%%%%%%%%%%%%%%%%%%%%%%%%%%%%%%

\documentclass[12pt]{report}
\usepackage[letterpaper]{geometry}
\usepackage[myheadings]{fullpage}
\usepackage{fancyhdr}
\usepackage{lastpage}
\usepackage{graphicx, wrapfig, subcaption, setspace, booktabs}
\usepackage[T1]{fontenc}
\usepackage[font=small, labelfont=bf]{caption}
\usepackage{fourier}
\usepackage[protrusion=true, expansion=true]{microtype}
\usepackage[english]{babel}
\usepackage{sectsty}
\usepackage{url, lipsum}
\usepackage{listings}
\usepackage{xcolor}

\lstset{
	numbers=left, 
	numberstyle= \tiny, 
	keywordstyle= \color{ blue!70},
	commentstyle= \color{red!50!green!50!blue!50}, 
	rulesepcolor= \color{ red!20!green!20!blue!20} ,
	escapeinside=``, % 英文分号中可写入中文
	xleftmargin=1em,xrightmargin=2em, aboveskip=1em,
	framexleftmargin=2em
} 

\geometry{left=0.8in,right=0.8in,top=1.5in,bottom=1.5in} 
\newcommand{\HRule}[1]{\rule{\linewidth}{#1}}
\onehalfspacing
\setcounter{tocdepth}{5}
\setcounter{secnumdepth}{5}

\setcounter{secnumdepth}{0}

%-------------------------------------------------------------------------------
% HEADER & FOOTER
%-------------------------------------------------------------------------------
\pagestyle{fancy}
\fancyhf{}
\setlength\headheight{15pt}
\fancyhead[L]{Microcontroller Programming}
\fancyfoot[C]{\thepage}

%-------------------------------------------------------------------------------
% TITLE PAGE
%-------------------------------------------------------------------------------

\begin{document}

\title{ \normalsize \textsc{BE WRITTEN REPORT}
		\\ [2.0cm]
		\HRule{0.5pt} \\
		\LARGE \textbf{\uppercase{Microcontroller Programming}}
		\HRule{2pt} \\ [0.5cm]
		\normalsize  \vspace*{5\baselineskip}}

\date{}

\author{
		\Large Siyi Xian \\
		Guided By - Prof. Randall D. Beer  \\
        Teammates - Zicheng Wang \& Nick de la Espriella\\
School of Informatics, Computing, and Engineering\\
Indiana Univeristy Bloomington\\ }

\maketitle
\newpage

\tableofcontents
\newpage

\section{Introduction}

The Microcontroller Programming project is a process in which a programmed program is placed into the Microcontroller via the RobotC platform, run in the Microcontroller and then return data to the RobotC platform. There are two difficulties in this project. The first is how to properly use the RobotC platform, and the second is how to use the existing API to accomplish the desired tasks in the C language. After completing the programming and compiling, we need to test the program in real time with the expected functionality and check if there are any areas worth improving. In Figure 1 below, it is a sample layout of RobotC programming environment.

\begin{figure}[htbp]
	\centering
	\includegraphics[width=0.5\textwidth]{RobotC.png}
	\caption{RobotC IDE Sample Layout}
\end{figure}

\section{Software Design}

Before designing the software, we need to first understand and be familiar with the RoboC operating environment. In RoboC we first need to turn the view to professional mode and change the debug mode to USB only. After the code is complete, we need to compile and download it into the Microcontroller and open the debug view. Then we can perform regular debugging.

\subsection{Hailstone Numbers}

\textit{Hailstone Numbers} are programs that require us to derive a string of numbers through a series of specific calculations. The main purpose is to test whether we are proficient in using RobotC's various runtime compilation functions. The code we write will actually run in the Microcontroller. The conclusions obtained will be returned to the computer screen in a view with debug mode. In programming, we first initialize an integer variable in the main program and start a $\mathtt{\textbf{while()}}$ loop. The loop's statement will be a think integer that cannot be $1$, because once the variable becomes $1$, it indicates the end of the program. In the loop, we first print the current variable on the debug view, then we use the helper function. The function of this function is to give a number and calculate the number of the next change in the queue. After the value of the next number has been passed, we assign it to the existing variable and proceed to the next iteration. In our helper function, we first use the $\mathtt{\textbf{if-else}}$ statement. The conditional statement is $\mathtt{\textbf{n\%2==1}}$, which means to interpret the parity of the currently given number. In the statement we bring different formulas into different forms. After completing the code debugging, everything is fine and fine-tuning the print format to complete this part.

\subsubsection{Code Snippets}
\lstset{language=C}
\begin{lstlisting}
/**************************************************************************
*           MAIN1
*  Pauses for 2 seconds then prints 'Hello World' to the debug stream
*  to display the debug stream:
*      -download the program to the cortex
*      -in the Robot menu select debugger
*      -go back to Robot; then go to the debugger window and select debug stream
*
***************************************************************************/

int hailstone(int n) {
	return n % 2 == 1 ? 3 * n + 1 : n / 2;
//	if (n % 2 == 1) {
//		return 3 * n + 1;
//	}
//	else {
//		return n / 2;
//	}
}

task main() {
	wait1Msec(2000);
	//writeDebugStream("Hello World");
	int n = 7;
	while (n != 1) {
		writeDebugStream("%d ", n);
		n = hailstone(n);
	}
	writeDebugStream("%d \n", n);
}
\end{lstlisting}

\subsection{A Light-Controlled Motor}
In the \textit{A Light-Controlled Motor} project, our goal is to make the SquareBot have different speeds of motion depending on the intensity of the light. The brighter the light, the faster the SquareBot travels. To accomplish this, I first need to turn the light sensor shown in Figure 2.
\begin{figure}[htbp]
	\centering
	\includegraphics[width=0.25\textwidth]{LightSensor.png}
	\caption{Light Sensor}
\end{figure}
In this part of programming, we are based on the template given on the course website and made changes. Since the given code is integrated with the default bumper, we not only need to change the speed of the default joystick control, we also need to change the speed that has changed since the bumper was triggered. In the programming, we will first use the built-in API to get the light sensor data. After investigating the API document, we found that the light sensor data is in the range $[0, 1000]$. A larger value means that the environment is darker. Since the maximum motor speed is $127$, we chose to set the speed to the range $[0.100]$. We choose to first subtract $1000$ from the light sensor data. The stronger the light, the larger the data. Then we choose to divide the data by $10$ to make the range at $[0,100]$. For the speed after the bumper triggers, we change the original maximum speed to the current processed lighting data. After connecting SquareBot to compile and debug, it was found that the speed of change can be more sensitive and achieve the desired effect. Print the current speed in the right place for debugging to complete this task.

\subsubsection{Code Snippets}
\lstset{language=C}
\begin{lstlisting}
#pragma config(Sensor, in1,    lightSensor,    sensorReflection)
#pragma config(Sensor, dgtl11, bump1,          sensorNone)
#pragma config(Sensor, dgtl12, bump2,          sensorNone)
#pragma config(Motor,  port2, leftMotor,  tmotorServoContinuousRotation, openLoop)
#pragma config(Motor,  port3, rightMotor, tmotorServoContinuousRotation, openLoop)
//*!!Code automatically generated by 'ROBOTC' configuration wizard               !!*//
	
/*******************************************************************************
*
*  This file is meant as an introduction and as a template for developing
*  programs to run on the Vex controller.
*
*******************************************************************************/
	
/*******************************************************************************
*
*	main
*
*******************************************************************************/
task main() {
	//  int bumper = 1;	              
	// normal bumper state = 1; when pressed = 0
	int lightValue;
	wait1Msec(2000);    					
	// give stuff time to turn on
	
	while(true) {
	
		// 1, read value of the light sensor
		lightValue = (1000 - SensorValue[lightSensor]) / 10;
		writeDebugStreamLine("%d", SensorValue[lightSensor]);
	
		//Joystick control of the motors
		motor[leftMotor]   =  lightValue;       // up = CW
		motor[rightMotor]  =  lightValue;       // up = CCW
	
		//If the bumper is press, automatically backup
		while(!SensorValue[bump1]) {
	
			// 2 - if light sensor > 0 < 10, spin motor slow
			writeDebugStreamLine("bumped-backing up");
	
			motor[leftMotor] = lightValue; //fullspeed backwards
			motor[rightMotor] = lightValue; //fullspeed backwards
			wait1Msec(1000);  // keep backing up for 1000ms (1 sec)
	
			motor[leftMotor]  = 0;  //stop motors
			motor[rightMotor] = 0;  //stop motors
			wait1Msec(1000);
		}
	
	writeDebugStreamLine("bumper is clear");
	}
}
	
\end{lstlisting}

\subsection{Anticipatory Obstacle Avoidance}
In the Anticipatory Obstacle Avoidance project, our goal is to have the SquareBot adjust its direction before it comes into contact with obstacles to avoid collisions with obstacles. In order to accomplish this goal, I first need to press the ultrasonic sensor, which actually is a tiny sonar, as shown in Figure 3.
\begin{figure}[htbp]
	\centering
	\includegraphics[width=0.25\textwidth]{Sonar.png}
	\caption{Ultrasonic Sensor }
\end{figure}
As with the previous task, this time the code is still changed based on the default code of the bumper. Since this feature is almost identical to the function of the bumper, we only need to make a few changes. The first thing that remains essential is getting the data. Based on the existing API, after matching the current in and out port, we easily obtained the data. Since the unit we selected is in centimeters, this time we don't need to process the data like the light sensor, just use it. After carefully studying the code, we found that we only need to change the judgment sentence in the loop statement. The previous judgment statement is to perform a turn when the bumper is triggered. This time we change it to change direction when the distance is less than $30cm$. In the default speed and cornering speed we set the number to $50$ and $100$ respectively to make it easier to control the SquareBot during testing. After several adjustments we set the turn time to $0.5$ seconds. The meaning is that when this speed, SquareBot can basically turn $90$ degrees.

\subsubsection{Code Snippets}

\lstset{language=C}
\begin{lstlisting}
#pragma config(Sensor, dgtl8,  ultraIn,        sensorSONAR_cm)
#pragma config(Sensor, dgtl11, bump1,          sensorNone)
#pragma config(Sensor, dgtl12, bump2,          sensorNone)
#pragma config(Motor,  port2,leftMotor, tmotorServoContinuousRotation, openLoop)
#pragma config(Motor,  port3,rightMotor,tmotorServoContinuousRotation, openLoop)
//*!!Code automatically generated by 'ROBOTC' configuration wizard               !!*//

/*******************************************************************************
*
*  This file is meant as an introduction and as a template for developing
*  programs to run on the Vex controller.
*
*******************************************************************************/

/*******************************************************************************
*
*	main
*
*******************************************************************************/
task main()
{
	//  int bumper = 1;	              
	// normal bumper state = 1; when pressed = 0

	wait1Msec(2000);    			// give stuff time to turn on

	while(true) {
		//Joystick control of the motors
		motor[leftMotor]   =  50;       // up = CW
		motor[rightMotor]  =  50;       // up = CCW

		writeDebugStreamLine("In: %d", SensorValue(ultraIn));
		//If the bumper is press, automatically backup
		while(SensorValue(ultraIn) <= 30) {
			writeDebugStreamLine("close to an obstacle");

			motor[leftMotor] = -100; //fullspeed backwards
			motor[rightMotor] = 100; //fullspeed backwards
			wait1Msec(500);  // keep backing up for 1000ms (1 sec)

			motor[leftMotor]  = 0;  //stop motors
			motor[rightMotor] = 0;  //stop motors
			wait1Msec(1000);
		}

		writeDebugStreamLine("way is clear");
	}
}

\end{lstlisting}

\section{Performance Evaluation}

First challenge we meet is that because C does not have $\mathtt{\textbf{string}}$, we can only use $\mathtt{\textbf{char[]}}$ for print out our data. Also we have a hard time when transfer integer to $\mathtt{\textbf{char[]}}$. However, finally we find out we can use some code like $\mathtt{\textbf{printf}}$ to format output.\\
In \textit{Hailstone Numbers} task, we made some mistake first. In programming, the first thing we try is to use the form of recursive to perform operations. We call the recursive equation in the main program. In the recursive function we first have a decision to end the statement to ensure that our recursive does not fall into an infinite loop. Next we are doing logarithmic operations. We used an $\mathtt{\textbf{if-else}}$ statement and used $\mathtt{\textbf{n\%2==1}}$ as the decision statement. After that we will print the value output in the debug view. Next, take this value into a new round of recursive function. Although this method can print all the answers correctly, we found that we violated the API design given by the program. For the return value of function, we did not give it correctly. After research, we found our help function, which is $\mathtt{\textbf{int}}$ $\mathtt{\textbf{hailstone(int}}$ $\mathtt{\textbf{n)}}$, just need to return the value of the next number. After asking professor and AIs, we finally make it perfectly. In our $\mathtt{\textbf{if-else}}$ in helper function, we use $\mathtt{\textbf{?}}$ and $\mathtt{\textbf{:}}$ to replace original one to make our code look more concise.\\
In the \textit{A Light-Controlled Motor} task, we spent a lot of time looking at the light sensor API usage and data acquisition in the document, but in the application we still have the error of not getting the data. Since we are using the analog signal, we follow the document content using $\mathtt{\textbf{anlg}}$ to get the data, but it is never available. After consulting the TA, we found that using RobotC, you can directly set the port data. In the end we used $\mathtt{\textbf{in}}$ as the key to get the data. \\
In the Anticipatory Obstacle Avoidance mission, our main problem is that the SquareBot turning angle is too large. The method we take is the value of the debugging speed and the turning time again and again, and finally the most suitable one is selected as the final value.

\section{Conclusion}

This project is the first time our team tried to use RobotC as our compilation platform to write code. Although we have more or less C language foundation, it is more challenging to practice on a brand new platform. And we need to be familiar with existing APIs in a short time and how to use them quickly and make good use of these used resources. We also need to note that because the document version of the different versions of the controller can not be perfectly matched with the code, we need to write the IDE as the basic. In the debugging process, we also need to pay attention to the appropriate output part of the data to facilitate debugging.

\end{document}

%-------------------------------------------------------------------------------
% SNIPPETS
%-------------------------------------------------------------------------------

%\begin{figure}[!ht]
%	\centering
%	\includegraphics[width=0.8\textwidth]{file_name}
%	\caption{}
%	\centering
%	\label{label:file_name}
%\end{figure}

%\begin{figure}[!ht]
%	\centering
%	\includegraphics[width=0.8\textwidth]{graph}
%	\caption{Blood pressure ranges and associated level of hypertension (American Heart Association, 2013).}
%	\centering
%	\label{label:graph}
%\end{figure}

%\begin{wrapfigure}{r}{0.30\textwidth}
%	\vspace{-40pt}
%	\begin{center}
%		\includegraphics[width=0.29\textwidth]{file_name}
%	\end{center}
%	\vspace{-20pt}
%	\caption{}
%	\label{label:file_name}
%\end{wrapfigure}

%\begin{wrapfigure}{r}{0.45\textwidth}
%	\begin{center}
%		\includegraphics[width=0.29\textwidth]{manometer}
%	\end{center}
%	\caption{Aneroid sphygmomanometer with stethoscope (Medicalexpo, 2012).}
%	\label{label:manometer}
%\end{wrapfigure}

%\begin{table}[!ht]\footnotesize
%	\centering
%	\begin{tabular}{cccccc}
%	\toprule
%	\multicolumn{2}{c} {Pearson's correlation test} & \multicolumn{4}{c} {Independent t-test} \\
%	\midrule	
%	\multicolumn{2}{c} {Gender} & \multicolumn{2}{c} {Activity level} & \multicolumn{2}{c} {Gender} \\
%	\midrule
%	Males & Females & 1st level & 6th level & Males & Females \\
%	\midrule
%	\multicolumn{2}{c} {BMI vs. SP} & \multicolumn{2}{c} {Systolic pressure} & \multicolumn{2}{c} {Systolic Pressure} \\
%	\multicolumn{2}{c} {BMI vs. DP} & \multicolumn{2}{c} {Diastolic pressure} & \multicolumn{2}{c} {Diastolic pressure} \\
%	\multicolumn{2}{c} {BMI vs. MAP} & \multicolumn{2}{c} {MAP} & \multicolumn{2}{c} {MAP} \\
%	\multicolumn{2}{c} {W:H ratio vs. SP} & \multicolumn{2}{c} {BMI} & \multicolumn{2}{c} {BMI} \\
%	\multicolumn{2}{c} {W:H ratio vs. DP} & \multicolumn{2}{c} {W:H ratio} & \multicolumn{2}{c} {W:H ratio} \\
%	\multicolumn{2}{c} {W:H ratio vs. MAP} & \multicolumn{2}{c} {\% Body fat} & \multicolumn{2}{c} {\% Body fat} \\
%	\multicolumn{2}{c} {} & \multicolumn{2}{c} {Height} & \multicolumn{2}{c} {Height} \\
%	\multicolumn{2}{c} {} & \multicolumn{2}{c} {Weight} & \multicolumn{2}{c} {Weight} \\
%	\multicolumn{2}{c} {} & \multicolumn{2}{c} {Heart rate} & \multicolumn{2}{c} {Heart rate} \\
%	\bottomrule
%	\end{tabular}
%	\caption{Parameters that were analysed and related statistical test performed for current study. BMI - body mass index; SP - systolic pressure; DP - diastolic pressure; MAP - mean arterial pressure; W:H ratio - waist to hip ratio.}
%	\label{label:tests}
%\end{table}